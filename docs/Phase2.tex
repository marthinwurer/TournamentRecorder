\documentclass[11pt]{article}

\usepackage[margin=.70in]{geometry}
\usepackage{tabulary}
\usepackage{verbatim}
\usepackage{array}
\usepackage{parskip}
\usepackage{graphicx}
\usepackage{float}
\usepackage{lineno}
\setlength\parskip{\baselineskip}
\setlength\parindent{0pt}
\newcommand{\mtg}{\textit{Magic$\colon$ the Gathering\textsuperscript{TM}} }

\linenumbers

\begin{document}
    \title{\mtg Tournament Recorder}
    \author{Team Tangent Tally \\ Benjamin Maitland \\ Jonathan Lo \\ Richard Ditullio \\ William Frey}
    \date{\today}

    \begin{titlepage}
        \maketitle
    \end{titlepage}

    \section{Overview}
    This application allows Tournament coordinators to record and track tournament data for a typical \mtg tournament.

    \subsection{Team Tangent Tally Members}
    \begin{center}
        \begin{tabular}{|l|l|}
            \hline
            \textbf{Roles} & \textbf{Name}\\
            \hline
            Team Leader & Benjamin Maitland\\
            \hline
            Design Engineer & Jonathan Lo\\
            \hline
            Test Engineer & Richard Ditullio\\
            \hline
            UI Engineer & William Frey\\
            \hline
        \end{tabular}
    \end{center}

    \newpage

    \section{Introduction}
    This section will discuss generic and detailed information about \mtg tournaments. The following definitions will be utilized later in the document.
    \subsection{Tournament Overview}
    \mtg tournaments are run as a multiple round Swiss tournament with a cut to a single elimination top 8. Pairing for each round is determined by player's current standings, where players with similar standings are paired to play against each other. At the end of a certain number of rounds, usually determined by the tournament's size, the top eight players by standing take part in a single-elimination finals bracket.


    \subsection{Definitions}
    \begin{table*}[h!]
        \centering
        \begin{tabulary}{\textwidth}{|p{10em}|p{30em}|}
            \hline
            \textbf{Terms} & \textbf{Definitions} \\
            \hline
            Game & A single play of the \mtg consisting of one or two players. A game is part of a \textit{Match}. A game results in a single winner.\\
            \hline
            Game Winner & A single player that is deemed to be the victor of the game.\\
            \hline
            Match & A series of games between at most two players. Matches will result in a winner or loser (in cases of a draw, no losers).\\
            \hline
            Match Pairing & A system to define the member(s) of the match. Match Pairings are calculated using the player's standings.\\
            \hline
            Match Score & The points given to the winner and loser, or in the case of a draw, both players.\\
            \hline
            Match Winner & The player that is first to win two games in a match. A match score of 3 is given to the player.\\
            \hline
            Match Loser & The player that was unable to win the match. A match score of 0 is given to the player.\\
            \hline
            Bye & A match with one player. The player is automatically declared the winner. A player can only play in one bye per tournament.\\
            \hline
            Round & A collection of matches. Rounds are defined as swiss or bracket.\\
            \hline
            Swiss Round & A subset of \textit{Round}. A Swiss Round is in which matches do not eliminate individuals from the proceeding rounds.\\
            \hline
            Bracket Round& A subset of \textit{Round}. A Bracket Round is in which matches eliminate individuals from the proceeding rounds.\\
            \hline
            Tournament & A collection of rounds, declaring a single winner.\\
            \hline
            Player Standing & A measurement of a player's comparative score amongst other players in the current tournament.\\
            \hline
            \end{tabulary}
        \caption{Terminology}
    \end{table*}

    \newpage

    \section{Software Design}
    \subsection{General Software Design}
    \begin{figure}[H]
        \fbox{\includegraphics[width=\textwidth]{HLR-State_Trans}}
        \centering
        \caption{State Transition Diagram}
    \end{figure}
    \subsection{Database Architecture}
    \begin{figure}[H]
        \fbox{\includegraphics[width=\textwidth]{HLR-Database_Struct}}
        \centering
        \caption{Database Design}
    \end{figure}
    \subsection{Software Architecture}
     \begin{figure}[H]
        \fbox{\includegraphics[width=\textwidth]{HLR-UML_Diagram}}
        \centering
        \caption{UML Diagram}
    \end{figure}

    \newpage

    \section{Requirements}
    \subsection{Software Requirements}
    \begin{itemize}
        \item The application shall be executable from a Windows or Linux operating system.
        \item The application shall be implemented using Python 3.5.
        \item The application shall utilize a MySQL-compatible Database to store, update, and access tournament records.
        \item The application shall implement the standard \mtg tournament structure and procedures.
    \end{itemize}

    \subsection{Software Design Requirements}
    \begin{itemize}
        \item The application shall implement an appropriate and usable interface for users.
        \item The application shall interface with the database, accessing and retrieving live tournament data and records.
        \item The application shall compute the necessary pairings for each round.
        \item The application shall require users to update necessary data, upon completion of a major state, such as match scores.
        \item The application shall compute necessary logic utilizing updated data within the database before an explicit state, such as match pairing for the next round.
        \item The application shall be testable with an automated and/or manual test(s).
        \item The application shall utilize appropriate design patterns, such as the Model-View-Controller design pattern.
    \end{itemize}

    \newpage

    \subsection{Software Implementation Requirements}
    \subsubsection{Database Implementation Requirements}

    \begin{itemize}
        \item The application shall store tournament information in the form:
            \begin{table*}[!hp]
            \centering
            \begin{tabulary}{\textwidth}{|p{9em}|p{7em}|p{8em}|p{18em}|}
                \hline
                \multicolumn{4}{|l|}{\textbf{Table}: \textbf{\textit{Tournaments}}}\\
                \hline
                \textbf{Attribute} & \textbf{Data Type} & \textbf{Descriptor} & \textbf{Comments}\\
                \hline
                ID & INT & PRIMARY KEY & An unique identifier generated by the database.\\
                \hline
                Name & VARCHAR(20) & PROVIDED & An identifier provided by the user, the name of the tournament.\\
                \hline
                Start Date & DATE & PROVIDED & Start date of tournament.\\
                \hline
                End Date & DATE & PROVIDED & End date of tournament.\\
                \hline
                Number of People & INT & PROVIDED & The number of participants.\\
                \hline
                Tournament Winner & INT & FOREIGN KEY & Reference to TOURNAMENT PLAYER,\\
                \hline
            \end{tabulary}
            \caption{Database Table: Tournaments}
            \end{table*}

            The \textbf{Table: tournament} shall be utilized for:
            \begin{itemize}
                \item tracking Tournament information
                \item providing reference to other tables to link toward
            \end{itemize}
        \item The application shall store round information in the form:
            \begin{table*}[!hp]
            \centering
            \begin{tabulary}{0.9\textwidth}{|p{7em}|p{6em}|p{8em}|p{20em}|}
                \hline
                \multicolumn{4}{|l|}{\textbf{Table}: \textbf{\textit{Rounds}}}\\
                \hline
                \textbf{Attribute} & \textbf{Data Type} & \textbf{Descriptor} & \textbf{Comments}\\
                \hline
                ID & INT & PRIMARY KEY & An unique identifier generated by the database.\\
                \hline
                Round Number & INT & GENERATED & An identifier generated by the program, the identifier of the round based on the tournament.\\
                \hline
                Start Time & TIME & GENERATED & A time for the round to begin.\\
                \hline
                End Time & TIME & GENERATED & A time for the round to end.\\
                \hline
                Tournament & INT & FOREIGN KEY & Reference to TOURNAMENT.\\
                \hline
                Type & ENUM \{Swiss, Bracket\} & GENERATED & The type of round.\\
                \hline
            \end{tabulary}
            \caption{Database Table: Rounds}
            \end{table*}

            The \textbf{Table: round} shall be utilized for:
            \begin{itemize}
                \item tracking Round information
                \item providing reference to other tables
            \end{itemize}

        \item The application shall store match information in the form:
            \begin{table*}[!hp]
            \centering
            \begin{tabulary}{0.9\textwidth}{|p{7em}|p{6em}|p{8em}|p{20em}|}
                \hline
                \multicolumn{4}{|l|}{\textbf{Table}: \textbf{\textit{Matches}}}\\
                \hline
                \textbf{Attribute} & \textbf{Data Type} & \textbf{Descriptor} & \textbf{Comments}\\
                \hline
                ID & INT & PRIMARY KEY & An unique identifier generated by the database.\\
                \hline
                Round Number & INT & FOREIGN KEY & Reference to ROUNDS.\\
                \hline
                Player1 & INT & FOREIGN KEY & Reference to TOURNAMENT PLAYERS.\\
                \hline
                Player2 & INT & FOREIGN KEY & Reference to TOURNAMENT PLAYERS.\\
                \hline
            \end{tabulary}
            \caption{Database Table: Matches}
            \end{table*}

            The \textbf{Table: t\_match} shall be utilized for:
            \begin{itemize}
                \item tracking Match information
                \item providing reference to other tables
            \end{itemize}

        \item The application shall store player information in the form:
            \begin{table*}[!hp]
            \centering
            \begin{tabulary}{0.9\textwidth}{|p{5em}|p{7em}|p{8em}|p{22em}|}
                \hline
                \multicolumn{4}{|l|}{\textbf{Table}: \textbf{\textit{Player}}}\\
                \hline
                \textbf{Attribute} & \textbf{Data Type} & \textbf{Descriptor} & \textbf{Comments}\\
                \hline
                ID & INT & PRIMARY KEY & An unique identifier generated by the database.\\
                \hline
                Name & VARCHAR(20) & PROVIDED & A given name of the player.\\
                \hline
                Wins & INT & GENERATED & Number of total wins by the player.\\
                \hline
                Losses & INT & GENERATED & Number of total losses by the player.\\
                \hline
                Draws & INT & GENERATED & Number of total draws by the player.\\
                \hline
            \end{tabulary}
            \caption{Database Table: Player}
            \end{table*}

            The \textbf{Table: player} shall be utilized for:
            \begin{itemize}
                \item tracking Player information
                \item tracking historical details of a Player entry.
            \end{itemize}

        \newpage

        \item The application shall store player tournament information in the form:
            \begin{table*}[!hp]
            \centering
            \begin{tabulary}{0.9\textwidth}{|p{8em}|p{6em}|p{8em}|p{20em}|}
                \hline
                \multicolumn{4}{|l|}{\textbf{Table}: \textbf{\textit{Tournament Players}}}\\
                \hline
                \textbf{Attribute} & \textbf{Data Type} & \textbf{Descriptor} & \textbf{Comments}\\
                \hline
                ID & INT & PRIMARY KEY & An unique identifier generated by the database.\\
                \hline
                Tournament ID & INT & FOREIGN KEY & Reference to TOURNAMENT.\\
                \hline
                Player ID & INT & FOREIGN KEY & Reference to Player.\\
                \hline
                Points & INT & GENERATED & Generated value of points that the player had gained within the tournament.\\
                \hline
                Tournament Wins & INT & GENERATED & Incrementing value that indicates wins by the player within the tournament.\\
                \hline
                Tournament Losses & INT & GENERATED & Incrementing value that indicates losses by the player within the tournament.\\
                \hline
                Tournament Draws & INT & GENERATED & Incrementing value that indicates draws by the player within the tournament.\\
                \hline
            \end{tabulary}
            \caption{Database Table: Tournament Players}
            \end{table*}

            The \textbf{Table: tournament\_player} shall be utilized for:
            \begin{itemize}
                \item tracking Tournament Players information.
                \item tracking historical details of a Tournament Players entry.
            \end{itemize}
    \end{itemize}
    \subsubsection{System Implementation Requirements}

    \subsubsection{Software Use Case Requirements}
        \begin{itemize}
            % insert use case tables
            \item Tournament Use Cases:
            \begin{itemize}
                \item The software shall allow a user to see a list of all tournaments.
                \item The software shall allow a user to create a new tournament.
                \item The software shall allow a user to view a list of players registered in a tournament.
                \item The software shall allow a user to register a player for a tournament.
                \item The software shall allow a user to remove a player from a tournament.
                \item The software shall allow a user to start a tournament.
                \item The software shall allow a user to view a list of rounds in a tournament.
                \item The software shall allow a user to end a tournament.
                \item The software shall allow a user to view tournaments that have finished.
            \end{itemize}

            % insert use case tables

            \item Round Use Cases:
            \begin{itemize}
                \item The software shall allow a user to start a round in a tournament.
                \item The software shall allow a user to end a round in a tournament.
                \item The software shall allow a user to generate pairings for a new round in the tournament.
                \item The software shall allow a user to view a list of matches in a round.
            \end{itemize}

            \item Match Use Cases:
            \begin{itemize}
                \item The software shall allow a user to report the results of a match.
                \item The software shall allow a user to change the results of a match.
                \item The software shall allow a user to view the results of a match.
            \end{itemize}

            \item Player Use Cases:
            \begin{itemize}
                \item The software shall allow a user to view the information of a player registered in a tournament.
                \item Te software shall allow a user to view a list of all players.
                \item The software shall allow a user to add a player.
                \item The software shall allow a user to view the information of a player.
            \end{itemize}
        \end{itemize}

    \section {Implementation}
    \subsection{Database Implementation}
        \begin{table*}[!hp]
        \centering
        \begin{tabulary}{\textwidth}{|p{9em}|p{7em}|p{8em}|p{18em}|}
            \hline
            \multicolumn{4}{|l|}{\textbf{Table}: \textbf{\texttt{tournament}}}\\
            \hline
            \textbf{Attribute} & \textbf{Data Type} & \textbf{Descriptor} & \textbf{Comments}\\
            \hline
            \texttt{id} & \texttt{INTEGER} & PRIMARY KEY & Auto incremented identifier generated by the database.\\
            \hline
            \texttt{name} & \texttt{VARCHAR(20)} & PROVIDED & An identifier provided by the user, the name of the tournament.\\
            \hline
            \texttt{max\_rounds} & \texttt{INTEGER} & PROVIDED & number of rounds in the tournmanet.\\
            \hline
            \texttt{start\_date} & \texttt{DATE} & PROVIDED & Start date of tournament.\\
            \hline
            \texttt{end\_date} & \texttt{DATE} & PROVIDED & End date of tournament.\\
            \hline
        \end{tabulary}
        \caption{Database Table: \texttt{tournament}}
        \end{table*}

        \begin{table*}[!hp]
        \centering
        \begin{tabulary}{0.9\textwidth}{|p{7em}|p{6em}|p{8em}|p{20em}|}
            \hline
            \multicolumn{4}{|l|}{\textbf{Table}: \textbf{\texttt{round}}}\\
            \hline
            \textbf{Attribute} & \textbf{Data Type} & \textbf{Descriptor} & \textbf{Comments}\\
            \hline
            \texttt{id} & \texttt{INTEGER} & PRIMARY KEY & Auto incremented identifier generated by the database.\\
            \hline
            \texttt{t\_id} & \texttt{INTEGER} & FOREIGN KEY & References to \texttt{tournament(id)}.\\
            \hline
            \texttt{number} & \texttt{INTEGER} & GENERATED & sequential number of round in tournament.\\
            \hline
            \texttt{start\_date} & \texttt{DATETIME} & GENERATED & A time for the round to begin.\\
            \hline
            \texttt{end\_time} & \texttt{DATETIME} & GENERATED & A time for the round to end.\\
            \hline
        \end{tabulary}
        \caption{Database Table: \texttt{round}}
        \end{table*}

        \begin{table*}[!hp]
        \centering
        \begin{tabulary}{0.9\textwidth}{|p{7em}|p{6em}|p{8em}|p{20em}|}
            \hline
            \multicolumn{4}{|l|}{\textbf{Table}: \textbf{\texttt{t\_match}}}\\
            \hline
            \textbf{Attribute} & \textbf{Data Type} & \textbf{Descriptor} & \textbf{Comments}\\
            \hline
            \texttt{id} & \texttt{INTEGER} & PRIMARY KEY & Auto incremented identifier generated by the database.\\
            \hline
            \texttt{r\_id} & \texttt{INTEGER} & FOREIGN KEY & Reference to \texttt{round(id)}.\\
            \hline
            \texttt{p1\_id} & \texttt{INTEGER} & FOREIGN KEY & Reference to \texttt{tournament\_player(id)}.\\
            \hline
            \texttt{p2\_id} & \texttt{INTEGER} & FOREIGN KEY & Reference to \texttt{tournament\_player(id)}.\\
            \hline
            \texttt{table\_number} & \texttt{INTEGER} & PROVIDED & assigned table number for the match.\\
            \hline
            \texttt{p1\_wins} & \texttt{INTEGER} & PROVIDED & player 1 wins.\\
            \hline
            \texttt{p2\_wins} & \texttt{INTEGER} & PROVIDED & player 2 wins.\\
            \hline
            \texttt{draws} & \texttt{INTEGER} & PROVIDED & number of draws in the match.\\
            \hline
        \end{tabulary}
        \caption{Database Table: \texttt{t\_match}}
        \end{table*}

        \begin{table*}[!hp]
        \centering
        \begin{tabulary}{0.9\textwidth}{|p{5em}|p{7em}|p{8em}|p{22em}|}
            \hline
            \multicolumn{4}{|l|}{\textbf{Table}: \textbf{\texttt{player}}}\\
            \hline
            \textbf{Attribute} & \textbf{Data Type} & \textbf{Descriptor} & \textbf{Comments}\\
            \hline
            \texttt{id} & \texttt{BIGINT} & PRIMARY KEY & An unique identifier generated by the database.\\
            \hline
            \texttt{name} & \texttt{VARCHAR(30)} & PROVIDED & A given name of the player.\\
            \hline
            \texttt{wins} & \texttt{INTEGER} & GENERATED & Number of total wins by the player.\\
            \hline
            \texttt{losses} & \texttt{INTEGER} & GENERATED & Number of total losses by the player.\\
            \hline
            \texttt{draws} & \texttt{INTEGER} & GENERATED & Number of total draws by the player.\\
            \hline
        \end{tabulary}
        \caption{Database Table: \texttt{player}}
        \end{table*}

        \begin{table*}[!hp]
        \centering
        \begin{tabulary}{0.9\textwidth}{|p{8em}|p{6em}|p{8em}|p{20em}|}
            \hline
            \multicolumn{4}{|l|}{\textbf{Table}: \textbf{\texttt{tournament\_player}}}\\
            \hline
            \textbf{Attribute} & \textbf{Data Type} & \textbf{Descriptor} & \textbf{Comments}\\
            \hline
            \texttt{id} & \texttt{INTEGER} & PRIMARY KEY & Auto incremented identifier generated by the database.\\
            \hline
            \texttt{t\_id} & \texttt{INTEGER} & FOREIGN KEY & Reference to \texttt{tournament(id)}.\\
            \hline
            \texttt{p\_id} & \texttt{BIGINT} & FOREIGN KEY & Reference to \texttt{player(id)}.\\
            \hline
            \texttt{dropped} & \texttt{INTEGER} & GENERATED & indicates whether they drop a tournament.\\
            \hline
        \end{tabulary}
        \caption{Database Table: \texttt{tournament\_player}}
        \end{table*}
    \newpage

    \subsection {API Implementation}
    \subsubsection{Method Summary}
        \begin{table*}[!hp]
            \centering
            \begin{tabulary}{0.9\textwidth}{|p{28em}|p{17em}|}
                \hline
                \textbf{Method Name} & \textbf{Description}\\
                \hline
                \texttt{addPlayer (Player ID, Tournament ID)} & adds a player to the tournament.\\
                \hline
                \texttt{createPlayer (DCI Number, Player Name)} & creates a player.\\
                \hline
                \texttt{createTournament (Name, Max Rounds)} & creates a tournament.\\
                \hline
                \texttt{finishRound (Round ID)} & finishes a round.\\
                \hline
                \texttt{generatePairing (Tournament ID)} & generates a new round and matches between players.\\
                \hline
                \texttt{getPlayer (Player ID)} & gets the player object.\\
                \hline
                \texttt{listPlayers ()} & gets the list of all players.\\
                \hline
                \texttt{listTournamentPlayers (Tournament ID)} & list all the players in the tournament.\\
                \hline
                \texttt{listTournaments (Sort By Field, Filter Types)} & gets a list of tournaments.\\
                \hline
                \texttt{matchList (Round ID)} & returns a list of matches for the current round.\\
                \hline
                \texttt{removePlayer (Player ID, Tournament ID)} & removes the player from the database.\\
                \hline
                \texttt{roundList (Tournament ID)} & returns list of rounds.\\
                \hline
                \texttt{searchPlayers (Partial Player Name)} & returns the player name of the ID.\\
                \hline
                \texttt{setMatchResults (Match ID, P1 Wins, P2 Wins, Draws)} & sets the match resu!hplt.\\
                \hline
                \texttt{startTournament (Tournament ID)} & initiates the tournament.\\
                \hline
            \end{tabulary}
            \caption{Method Summary}
        \end{table*}

    \subsubsection {Method Details}
        \begin{table*}[!hp]
            \centering
            \begin{tabulary}{0.9\textwidth}{|p{8em}|p{8em}|p{28em}|}
                \hline
                \multicolumn{3}{|l|}{ \texttt{addPlayer ( p\_id, t\_id )}}\\
                \hline
                \textbf{Parameter} & \textbf{Data Type} & \textbf{Descriptor}\\
                \hline
                \texttt{p\_id} & \texttt{Integer} & the official given id of the player.\\
                \hline
                \texttt{t\_id} & \texttt{Integer} & the tournament ID.\\
                \hline
                \multicolumn{3}{|l|}{adds the player of \texttt{p\_id} to the tournament \texttt{t\_id}}\\
                \hline
            \end{tabulary}
            \caption{\texttt{addPlayer()} method }
        \end{table*}
        \begin{table*}[!hp]
            \centering
            \begin{tabulary}{0.9\textwidth}{|p{8em}|p{8em}|p{28em}|}
                \hline
                \multicolumn{3}{|l|}{ \texttt{createPlayer ( DCI, name )}}\\
                \hline
                \textbf{Parameter} & \textbf{Data Type} & \textbf{Descriptor}\\
                \hline
                \texttt{DCI} & \texttt{Integer} & the Player's DCI number (Unique Identifier).\\
                \hline
                \texttt{name} & \texttt{String} & the name of the player.\\
                \hline
                \multicolumn{3}{|l|}{creates a player with the given \texttt{name} and \texttt{DCI}}\\
                \hline
            \end{tabulary}
            \caption{\texttt{createPlayer()} method }
        \end{table*}
        \begin{table*}[!hp]
            \centering
            \begin{tabulary}{0.9\textwidth}{|p{8em}|p{8em}|p{28em}|}
                \hline
                \multicolumn{3}{|l|}{ \texttt{createTournament ( name, max\_rounds )}}\\
                \hline
                \textbf{Parameter} & \textbf{Data Type} & \textbf{Descriptor}\\
                \hline
                \texttt{name} & \texttt{String} & name of the tournament.\\
                \hline
                \texttt{max\_rounds} & \texttt{Integer} & the maximum number of rounds in the tournament.\\
                \hline
                \multicolumn{3}{|l|}{creates a tournament with the given \texttt{name} and \texttt{max\_rounds}}\\
                \hline
            \end{tabulary}
            \caption{\texttt{createTournament()} method }
        \end{table*}
        \begin{table*}[!hp]
            \centering
            \begin{tabulary}{0.9\textwidth}{|p{8em}|p{8em}|p{28em}|}
                \hline
                \multicolumn{3}{|l|}{ \texttt{finishRound ( r\_id )}}\\
                \hline
                \textbf{Parameter} & \textbf{Data Type} & \textbf{Descriptor}\\
                \hline
                \texttt{r\_id} & \texttt{Integer} & the round ID.\\
                \hline
                \multicolumn{3}{|l|}{sets the round as completed.}\\
                \hline
            \end{tabulary}
            \caption{\texttt{finishRound()} method }
        \end{table*}
        \begin{table*}[!hp]
            \centering
            \begin{tabulary}{0.9\textwidth}{|p{8em}|p{8em}|p{28em}|}
                \hline
                \multicolumn{3}{|l|}{ \texttt{generatePairing ( t\_id )}}\\
                \hline
                \textbf{Parameter} & \textbf{Data Type} & \textbf{Descriptor}\\
                \hline
                \texttt{t\_id} & \texttt{Integer} & the tournament ID.\\
                \hline
                \multicolumn{3}{|l|}{generates a round and the matches of the round.}\\
                \hline
            \end{tabulary}
            \caption{\texttt{generatePairing()} method }
        \end{table*}
        \begin{table*}[!hp]
            \centering
            \begin{tabulary}{0.9\textwidth}{|p{8em}|p{8em}|p{28em}|}
                \hline
                \multicolumn{3}{|l|}{ \texttt{getPlayers ( p\_id )}}\\
                \hline
                \textbf{Parameter} & \textbf{Data Type} & \textbf{Descriptor}\\
                \hline
                \texttt{p\_id} & \texttt{Integer} & the id of the player.\\
                \hline
                \multicolumn{3}{|l|}{returns the player object of the given \texttt{p\_id}}\\
                \hline
            \end{tabulary}
            \caption{\texttt{getPlayer()} method }
        \end{table*}
        \begin{table*}[!hp]
            \centering
            \begin{tabulary}{0.9\textwidth}{|p{8em}|p{8em}|p{28em}|}
                \hline
                \multicolumn{3}{|l|}{ \texttt{listTournament ( sort\_on, filter\_types )}}\\
                \hline
                \textbf{Parameter} & \textbf{Data Type} & \textbf{Descriptor}\\
                \hline
                \texttt{sort\_on} & \texttt{List} & a list of attributes to sort on.\\
                \hline
                \texttt{filter\_types} & \texttt{Character} & flags to filter the returning list by. Flags consist of waiting, started and finished.\\
                \hline
                \multicolumn{3}{|l|}{returns a list of tournaments objects with defined attributes}\\
                \hline
            \end{tabulary}
            \caption{\texttt{listTournament()} method }
        \end{table*}
        \begin{table*}[!hp]
            \centering
            \begin{tabulary}{0.9\textwidth}{|p{8em}|p{8em}|p{28em}|}
                \hline
                \multicolumn{3}{|l|}{ \texttt{listTournamentPlayers ( t\_id )}}\\
                \hline
                \textbf{Parameter} & \textbf{Data Type} & \textbf{Descriptor}\\
                \hline
                \texttt{t\_id} & \texttt{Integer} & the tournament id.\\
                \hline
                \multicolumn{3}{|l|}{returns a list of tournaments players}\\
                \hline
            \end{tabulary}
            \caption{\texttt{listTournamentPlayers()} method }
        \end{table*}
        \begin{table*}[!hp]
            \centering
            \begin{tabulary}{0.9\textwidth}{|p{8em}|p{8em}|p{28em}|}
                \hline
                \multicolumn{3}{|l|}{ \texttt{matchList ( r\_id )}}\\
                \hline
                \textbf{Parameter} & \textbf{Data Type} & \textbf{Descriptor}\\
                \hline
                \texttt{r\_id} & \texttt{Integer} & the round ID.\\
                \hline
                \multicolumn{3}{|l|}{returns a list of matches for the current round..}\\
                \hline
            \end{tabulary}
            \caption{\texttt{matchList()} method }
        \end{table*}
        \begin{table*}[!hp]
            \centering
            \begin{tabulary}{0.9\textwidth}{|p{8em}|p{8em}|p{28em}|}
                \hline
                \multicolumn{3}{|l|}{ \texttt{removePlayer ( p\_id, t\_id )}}\\
                \hline
                \textbf{Parameter} & \textbf{Data Type} & \textbf{Descriptor}\\
                \hline
                \texttt{p\_id} & \texttt{Integer} & the player id.\\
                \hline
                \texttt{t\_id} & \texttt{Integer} & the tournament ID.\\
                \hline
                \multicolumn{3}{|l|}{adds the player of \texttt{p\_id} to the tournament \texttt{t\_id}}\\
                \hline
            \end{tabulary}
            \caption{\texttt{removePlayer()} method }
        \end{table*}
        \begin{table*}[!hp]
            \centering
            \begin{tabulary}{0.9\textwidth}{|p{8em}|p{8em}|p{28em}|}
                \hline
                \multicolumn{3}{|l|}{ \texttt{roundList ( t\_id )}}\\
                \hline
                \textbf{Parameter} & \textbf{Data Type} & \textbf{Descriptor}\\
                \hline
                \texttt{t\_id} & \texttt{Integer} & the tournament ID.\\
                \hline
                \multicolumn{3}{|l|}{gets the list of rounds part of the tournament.}\\
                \hline
            \end{tabulary}
            \caption{\texttt{roundList()} method }
        \end{table*}
        \begin{table*}[!hp]
            \centering
            \begin{tabulary}{0.9\textwidth}{|p{8em}|p{8em}|p{28em}|}
                \hline
                \multicolumn{3}{|l|}{ \texttt{searchPlayers ( partial\_name )}}\\
                \hline
                \textbf{Parameter} & \textbf{Data Type} & \textbf{Descriptor}\\
                \hline
                \texttt{partial\_name} & \texttt{String} & partial string name.\\
                \hline
                \multicolumn{3}{|l|}{searches for a player with the given \texttt{partial\_name} and returns players name}\\
                \hline
            \end{tabulary}
            \caption{\texttt{searchPlayers()} method }
        \end{table*}
        \begin{table*}[!hp]
            \centering
            \begin{tabulary}{0.9\textwidth}{|p{8em}|p{8em}|p{28em}|}
                \hline
                \multicolumn{3}{|l|}{ \texttt{setMatchTournament ( m\_id, p1\_wins, p2\_wins, draws )}}\\
                \hline
                \textbf{Parameter} & \textbf{Data Type} & \textbf{Descriptor}\\
                \hline
                \texttt{m\_id} & \texttt{Integer} & the match id.\\
                \hline
                \texttt{p1\_wins} & \texttt{Integer} & player 1's wins.\\
                \hline
                \texttt{p2\_wins} & \texttt{Integer} & player 2's wins.\\
                \hline
                \texttt{draws} & \texttt{Integer} & number of draws.\\
                \hline
                \multicolumn{3}{|l|}{sets the scoring for the match between player 1 and player 2.}\\
                \hline
            \end{tabulary}
            \caption{\texttt{setMatchTournament()} method }
        \end{table*}
        \begin{table*}[!hp]
            \centering
            \begin{tabulary}{0.9\textwidth}{|p{8em}|p{8em}|p{28em}|}
                \hline
                \multicolumn{3}{|l|}{ \texttt{startTournament ( t\_id )}}\\
                \hline
                \textbf{Parameter} & \textbf{Data Type} & \textbf{Descriptor}\\
                \hline
                \texttt{t\_id} & \texttt{Integer} & the tournament ID.\\
                \hline
                \multicolumn{3}{|l|}{starts the specified tournament. This enables processing of the tournament.}\\
                \hline
            \end{tabulary}
            \caption{\texttt{startTournament()} method }
        \end{table*}

\end{document}
