\documentclass[11pt]{article}

\usepackage[margin=.80in]{geometry}
\usepackage{tabulary}
\usepackage{verbatim}
\usepackage{array}
\usepackage{parskip}
\usepackage{graphicx}
\usepackage{float}
\usepackage{lineno}
\usepackage[utf8]{inputenc}
\usepackage[english]{babel}
\usepackage{fancyhdr}
\usepackage{imakeidx}
\usepackage{hyperref}
\usepackage{listings}
\usepackage{color}

\setlength\parskip{\baselineskip}
\setlength\parindent{0pt}

\newcommand{\mtg}{\textit{Magic$\colon$ the Gathering\textsuperscript{TM}} }
\newcommand{\rheadDetails}{\textbf{Version C}: Updated \textit{\today}}

\linenumbers
% \pagenumbering{}
\makeindex

\usepackage{hyperref}
\hypersetup{
    colorlinks=true,
    linkcolor=black,
    filecolor=magenta,
    urlcolor=blue,
}

\begin{document}
    \title{\textbf{\mtg Tournament Recorder}}
    \author{Team Tangent Tally}
    \date{\textbf{November 2016}}

    \begin{titlepage}
        \LARGE
        \maketitle
        \begin{center}
            \normalsize
                \textbf{Change History}\\
            \small
                \rheadDetails\\
                \textbf{Verion B}: Updated \textit{October 28, 2016}\\
                \textbf{Verion A}: Updated \textit{September 30, 2016}\\
        \end{center}
    \end{titlepage}

    \pagestyle{fancy}
    \fancyhf{}
    \fancyhead[LE,RO]{\rheadDetails}
    \fancyhead[RE,LO]{Tangent Tally}
    \fancyfoot[CE,CO]{\leftmark}
    \fancyfoot[LE,RO]{\thepage}

    \tableofcontents
    \newpage

    \section{Overview}

        \subsection{Product Intentions}
        This application allows Tournament coordinators to record and track tournament data for a typical \mtg tournament.

        \subsection{Team Tangent Tally Members}
        \begin{center}
            \begin{tabular}{|l|l|}
                \hline
                \textbf{Roles} & \textbf{Name}\\
                \hline
                Team Leader & Benjamin Maitland\\
                \hline
                Design Engineer & Jonathan Lo\\
                \hline
                Test Engineer & Richard Ditullio\\
                \hline
                UI Engineer & William Frey\\
                \hline
            \end{tabular}
        \end{center}

        \newpage

    \section{Introduction}
        This section will discuss generic and detailed information about \mtg tournaments. The following definitions will be utilized later in the document.

        \subsection{Tournament Overview}
        \mtg tournaments are run as a multiple round Swiss tournament with a cut to a single elimination top 8. Pairing for each round is determined by player's current standings, where players with similar standings are paired to play against each other. At the end of a certain number of rounds, usually determined by the tournament's size, the top eight players by standing take part in a single-elimination finals bracket.


        \subsection{Definitions}
        \begin{table*}[h!]
            \centering
            \begin{tabulary}{\textwidth}{|p{10em}|p{30em}|}
                \hline
                \textbf{Terms} & \textbf{Definitions} \\
                \hline
                Game & A single play of the \mtg consisting of one or two players. A game is part of a \textit{Match}. A game results in a single winner.\\
                \hline
                Game Winner & A single player that is deemed to be the victor of the game.\\
                \hline
                Match & A series of games between at most two players. Matches will result in a winner or loser (in cases of a draw, no losers).\\
                \hline
                Match Pairing & A system to define the member(s) of the match. Match Pairings are calculated using the player's standings.\\
                \hline
                Match Score & The points given to the winner and loser, or in the case of a draw, both players.\\
                \hline
                Match Winner & The player that is first to win two games in a match. A match score of 3 is given to the player.\\
                \hline
                Match Loser & The player that was unable to win the match. A match score of 0 is given to the player.\\
                \hline
                Bye & A match with one player. The player is automatically declared the winner. A player can only play in one bye per tournament.\\
                \hline
                Round & A collection of matches. Rounds are defined as swiss or bracket.\\
                \hline
                Swiss Round & A subset of \textit{Round}. A Swiss Round is in which matches do not eliminate individuals from the proceeding rounds.\\
                \hline
                Bracket Round& A subset of \textit{Round}. A Bracket Round is in which matches eliminate individuals from the proceeding rounds.\\
                \hline
                Tournament & A collection of rounds, declaring a single winner.\\
                \hline
                Player Standing & A measurement of a player's comparative score amongst other players in the current tournament.\\
                \hline
                \end{tabulary}
            \caption{Terminology}
        \end{table*}

    \newpage

    \newpage

    \section{Software Design}
        This section outlines the product design, infrastructure design and software architecture.
        \subsection{Software Design}
            \begin{figure}[H]
                \fbox{\includegraphics[width=\textwidth]{HLR-State_Trans}}
                \caption{State Transition Diagram}
            \end{figure}
            \subsection{Database Architecture}
            \begin{figure}[H]
                \fbox{\includegraphics[width=\textwidth]{HLR-Database_Struct}}
                \caption{Database Design}
            \end{figure}
            \subsection{Software Architecture}
             \begin{figure}[H]
                \fbox{\includegraphics[width=\textwidth]{HLR-UML_Diagram}}
                \caption{UML Diagram}
            \end{figure}

    \newpage

    \newpage

    \section{Requirements}
    \subsection{Software Requirements}
    \begin{itemize}
        \item The application shall be executable from a Windows or Linux operating system.
        \item The application shall be implemented using Python 3.5.
        \item The application shall be implemented using Tkinter 8.6.
        \item The application shall utilize a MySQL-compatible Database to store, update, and access tournament records.
        \item The application shall implement the standard \mtg tournament structure and procedures.
    \end{itemize}

    \subsection{Software Design Requirements}
    \begin{itemize}
        \item The application shall implement an appropriate and usable interface for users.
        \item The application shall interface with the database, accessing and retrieving live tournament data and records.
        \item The application shall compute the necessary pairings for each round.
        \item The application shall require users to update necessary data, upon completion of a major state, such as match scores.
        \item The application shall compute necessary logic utilizing updated data within the database before an explicit state, such as match pairing for the next round.
        \item The application shall be testable with an automated and/or manual test(s).
        \item The application shall utilize appropriate design patterns, such as the Model-View-Controller design pattern.
    \end{itemize}

    \newpage

    \subsection{Software Implementation Requirements}
        \subsubsection{Database Implementation Requirements}
            The \textbf{Table: \texttt{player}} shall be utilized for:
            \begin{itemize}
                \item tracking Player information
                \item tracking historical details of a Player entry.
            \end{itemize}

            The \textbf{Table: \texttt{tournament}} shall be utilized for:
            \begin{itemize}
                \item tracking tournament information
                \item providing reference for other table links.
            \end{itemize}

            The \textbf{Table: \texttt{round}} shall be utilized for:
            \begin{itemize}
                \item tracking Round information
                \item providing reference to other tables
            \end{itemize}


            The \textbf{Table: \texttt{t\_match}} shall be utilized for:
            \begin{itemize}
                \item tracking Match information for rounds.
                \item providing reference to other tables
            \end{itemize}

            The \textbf{Table: tournament\_player} shall be utilized for:
            \begin{itemize}
                \item tracking Tournament Players information.
                \item tracking historical details of a Tournament Players entry.
            \end{itemize}

        \subsubsection{System Implementation Requirements}
            There are no known system implementation requirements.

        \newpage

        \subsubsection{Software Use Case Requirements}
            \begin{itemize}
            % insert use case tables
            \item Tournament Use Cases:
            \begin{itemize}
                \item The software shall allow a user to see a list of all tournaments.
                \item The software shall allow a user to create a new tournament.
                \item The software shall allow a user to view a list of players registered in a tournament.
                \item The software shall allow a user to register a player for a tournament.
                \item The software shall allow a user to remove a player from a tournament.
                \item The software shall allow a user to start a tournament.
                \item The software shall allow a user to view a list of rounds in a tournament.
                \item The software shall allow a user to end a tournament.
                \item The software shall allow a user to view tournaments that have finished.
            \end{itemize}

            % insert use case tables

            \item Round Use Cases:
            \begin{itemize}
                \item The software shall allow a user to start a round in a tournament.
                \item The software shall allow a user to end a round in a tournament.
                \item The software shall allow a user to generate pairings for a new round in the tournament.
                \item The software shall allow a user to view a list of matches in a round.
            \end{itemize}

            \item Match Use Cases:
            \begin{itemize}
                \item The software shall allow a user to report the results of a match.
                \item The software shall allow a user to change the results of a match.
                \item The software shall allow a user to view the results of a match.
            \end{itemize}

            \item Player Use Cases:
            \begin{itemize}
                \item The software shall allow a user to view the information of a player registered in a tournament.
                \item Te software shall allow a user to view a list of all players.
                \item The software shall allow a user to add a player.
                \item The software shall allow a user to view the information of a player.
            \end{itemize}
        \end{itemize}

    \newpage

    \section {Implementation}
        \subsection{Database Implementation}
        \begin{table*}[!hp]
            \centering
            \begin{tabulary}{0.9\textwidth}{|p{6em}|p{6em}|p{6em}|p{23em}|}
                \hline
                \multicolumn{4}{|l|}{\textbf{Table}: \textbf{\texttt{player}}}\\
                \hline
                \textbf{Attribute}  & \textbf{Data Type}    & \textbf{Descriptor}   & \textbf{Comments}\\
                \hline
                \texttt{id}         & \texttt{BIGINT(20)}   & \texttt{PRIMARY KEY}  & An unique identifier generated by the database.\\
                \hline
                \texttt{name}       & \texttt{VARCHAR(30)}  & \texttt{PROVIDED}     & A given name of the player.\\
                \hline
                \texttt{wins}       & \texttt{INT(11)}      & \texttt{GENERATED}    & Number of total wins by the player.\\
                \hline
                \texttt{losses}     & \texttt{INT(11)}      & \texttt{GENERATED}    & Number of total losses by the player.\\
                \hline
                \texttt{draws}      & \texttt{INT(11)}      & \texttt{GENERATED}    & Number of total draws by the player.\\
                \hline
            \end{tabulary}
            \caption{Database Table: \texttt{player}}
        \end{table*}

        \begin{table*}[!hp]
            \centering
            \begin{tabulary}{0.9\textwidth}{|p{6em}|p{6em}|p{6em}|p{23em}|}
                \hline
                \multicolumn{4}{|l|}{\textbf{Table}: \textbf{\texttt{round}}}\\
                \hline
                \textbf{Attribute}      & \textbf{Data Type}    & \textbf{Descriptor}   & \textbf{Comments}\\
                \hline
                \texttt{id}             & \texttt{INT(11)}      & \texttt{PRIMARY KEY}  & Auto incremented identifier generated by the database.\\
                \hline
                \texttt{t\_id}          & \texttt{INT(11)}      & \texttt{FOREIGN KEY}  & References to \texttt{tournament(id)}.\\
                \hline
                \texttt{number}         & \texttt{INT(11)}      & \texttt{GENERATED}    & sequential number of round in tournament.\\
                \hline
                \texttt{start\_date}    & \texttt{DATETIME}     & \texttt{GENERATED}    & A time for the round to begin.\\
                \hline
                \texttt{end\_time}      & \texttt{DATETIME}     & \texttt{GENERATED}    & A time for the round to end.\\
                \hline
            \end{tabulary}
            \caption{Database Table: \texttt{round}}
        \end{table*}

        \begin{table*}[!hp]
            \centering
            \begin{tabulary}{0.9\textwidth}{|p{6em}|p{6em}|p{6em}|p{23em}|}
                \hline
                \multicolumn{4}{|l|}{\textbf{Table}: \textbf{\texttt{t\_match}}}\\
                \hline
                \textbf{Attribute}      & \textbf{Data Type}    & \textbf{Descriptor}   & \textbf{Comments}\\
                \hline
                \texttt{id}             & \texttt{INT(11)}      & \texttt{PRIMARY KEY}  & Auto incremented identifier generated by the database.\\
                \hline
                \texttt{r\_id}          & \texttt{INT(11)}      & \texttt{FOREIGN KEY}  & Reference to \texttt{round(id)}.\\
                \hline
                \texttt{p1\_id}         & \texttt{INT(11)}      & \texttt{FOREIGN KEY}  & Reference to \texttt{tournament\_player(id)}.\\
                \hline
                \texttt{p2\_id}         & \texttt{INT(11)}      & \texttt{FOREIGN KEY}  & Reference to \texttt{tournament\_player(id)}.\\
                \hline
                \texttt{table\_number}  & \texttt{INT(11)}      & \texttt{PROVIDED}     & assigned table number for the match.\\
                \hline
                \texttt{p1\_wins}       & \texttt{INT(11)}      & \texttt{PROVIDED}     & player 1 wins.\\
                \hline
                \texttt{p2\_wins}       & \texttt{INT(11)}      & \texttt{PROVIDED}     & player 2 wins.\\
                \hline
                \texttt{draws}          & \texttt{INT(11)}      & \texttt{PROVIDED}     & number of draws in the match.\\
                \hline
            \end{tabulary}
            \caption{Database Table: \texttt{t\_match}}
        \end{table*}

        \begin{table*}[!hp]
            \centering
            \begin{tabulary}{\textwidth}{|p{6em}|p{6em}|p{6em}|p{23em}|}
                \hline
                \multicolumn{4}{|l|}{\textbf{Table}: \textbf{\texttt{tournament}}}\\
                \hline
                \textbf{Attribute}      & \textbf{Data Type}    & \textbf{Descriptor}   & \textbf{Comments}\\
                \hline
                \texttt{id}             & \texttt{INT(11)}      & \texttt{PRIMARY KEY}  & Auto incremented identifier generated by the database.\\
                \hline
                \texttt{name}           & \texttt{VARCHAR(20)}  & \texttt{PROVIDED}     & An identifier provided by the user, the name of the tournament.\\
                \hline
                \texttt{max\_rounds}    & \texttt{INT(11)}      & \texttt{PROVIDED}     & number of rounds in the tournament.\\
                \hline
                \texttt{start\_date}    & \texttt{DATE}         & \texttt{PROVIDED}     & Start date of tournament.\\
                \hline
                \texttt{end\_date}      & \texttt{DATE}         & \texttt{PROVIDED}     & End date of tournament.\\
                \hline
            \end{tabulary}
            \caption{Database Table: \texttt{tournament}}
        \end{table*}

        \newpage

        \begin{table*}[!hp]
            \centering
            \begin{tabulary}{0.9\textwidth}{|p{6em}|p{6em}|p{6em}|p{23em}|}
                \hline
                \multicolumn{4}{|l|}{\textbf{Table}: \textbf{\texttt{tournament\_player}}}\\
                \hline
                \textbf{Attribute}  & \textbf{Data Type}    & \textbf{Descriptor}   & \textbf{Comments}\\
                \hline
                \texttt{id}         & \texttt{INT(11)}      & \texttt{PRIMARY KEY}  & An unique identifier generated by the database.\\
                \hline
                \texttt{t\_id}      & \texttt{INT(11)}      & \texttt{FOREIGN KEY}  & Reference to \texttt{tournament(id)}.\\
                \hline
                \texttt{p\_id}      & \texttt{BIGINT(20)}   & \texttt{FOREIGN KEY}  & Reference to \texttt{tournament\_player(id)}.\\
                \hline
                \texttt{dropped}    & \texttt{INT(11)}      & \texttt{GENERATED}    & Indicates if the player was dropped from the tournament.\\
                \hline
            \end{tabulary}
            \caption{Database Table: \texttt{tournament\_player}}
        \end{table*}

        \subsection {API Implementation}
            \subsubsection{Method Summary}
                \begin{table*}[!hp]
                    \centering
                    \begin{tabulary}{0.9\textwidth}{|p{27em}|p{16em}|}
                        \hline
                        \textbf{Method Name} & \textbf{Description}\\
                        \hline
                        \texttt{addPlayer (Player ID, Tournament ID)}                   & adds a player to the tournament.\\
                        \hline
                        \texttt{createPlayer (DCI Number, Player Name)}                 & creates a player.\\
                        \hline
                        \texttt{createTournament (Name, Max Rounds)}                    & creates a tournament.\\
                        \hline
                        \texttt{finishRound (Round ID)}                                 & finishes a round.\\
                        \hline
                        \texttt{generatePairing (Tournament ID)}                        & generates a new round and matches between players.\\
                        \hline
                        \texttt{getPlayer (Player ID)}                                  & gets the specified player from Table: \texttt{Players}.\\
                        \hline
                        \texttt{getTournamentPlayer (tourn. player ID)}                 & get the specified player from Table: \texttt{Tournament Players}.\\
                        \hline
                        \texttt{listActiveTournamentPlayers (Tournament ID)}            & list all players actively participating in the tournament.\\
                        \hline
                        \texttt{listPlayers ()}                                         & gets the list of all players.\\
                        \hline
                        \texttt{listTournamentPlayers (Tournament ID)}                  & list all the players in the tournament.\\
                        \hline
                        \texttt{listTournaments (Sort By Field, Filter Types)}          & gets a list of tournaments.\\
                        \hline
                        \texttt{matchList (Round ID)}                                   & returns a list of matches for the current round.\\
                        \hline
                        \texttt{removePlayer (Tourn. Player ID, Tournament ID)}         & removes the player from the database.\\
                        \hline
                        \texttt{roundList (Tournament ID)}                              & returns list of rounds.\\
                        \hline
                        \texttt{searchPlayers (Partial Player Name)}                    & returns the player name of the ID.\\
                        \hline
                        \texttt{setMatchResults (Match ID, P1 Wins, P2 Wins, Draws)}    & sets the match resu!hplt.\\
                        \hline
                        \texttt{startTournament (Tournament ID)}                        & initiates the tournament.\\
                        \hline
                    \end{tabulary}
                    \caption{Method Summary}
                \end{table*}

            \newpage

            \subsubsection {Method Details}
                \begin{table*}[!hp]
                    \centering
                    \begin{tabulary}{0.9\textwidth}{|p{8em}|p{8em}|p{28em}|}
                        \hline
                        \multicolumn{3}{|l|}{\texttt{addPlayer ( p\_id, t\_id )}}\\
                        \hline
                        \textbf{Parameter} & \textbf{Data Type} & \textbf{Descriptor}\\
                        \hline
                        \texttt{p\_id} & \texttt{Integer} & the official given id of the player.\\
                        \hline
                        \texttt{t\_id} & \texttt{Integer} & the tournament ID.\\
                        \hline
                        \multicolumn{3}{|l|}{adds the player of \texttt{p\_id} to the tournament \texttt{t\_id}}.\\
                        \hline
                        \multicolumn{3}{|l|}{\textbf{Returns:}}\\
                        \hline
                        \multicolumn{3}{|l|}{\texttt{\{"outcome" : True\}} on success.}\\
                        \hline
                        \multicolumn{3}{|l|}{\texttt{\{"outcome" : False, "reason" : \dots\}} on failure.}\\
                        \hline
                    \end{tabulary}
                    \caption{\texttt{addPlayer()} method }
                \end{table*}

                \begin{table*}[!hp]
                    \centering
                    \begin{tabulary}{0.9\textwidth}{|p{8em}|p{8em}|p{28em}|}
                        \hline
                        \multicolumn{3}{|l|}{ \texttt{createPlayer ( DCI, name )}}\\
                        \hline
                        \textbf{Parameter} & \textbf{Data Type} & \textbf{Descriptor}\\
                        \hline
                        \texttt{DCI} & \texttt{Integer} & the Player's DCI number (Unique Identifier).\\
                        \hline
                        \texttt{name} & \texttt{String} & the name of the player.\\
                        \hline
                        \multicolumn{3}{|l|}{creates a player with the given \texttt{name} and \texttt{DCI}}\\
                        \hline
                        \multicolumn{3}{|l|}{\textbf{Returns:}}\\
                        \hline
                        \multicolumn{3}{|l|}{\texttt{\{"outcome" : True\}} on success.}\\
                        \hline
                        \multicolumn{3}{|l|}{\texttt{\{"outcome" : False, "reason" : \dots\}} on failure.}\\
                        \hline
                    \end{tabulary}
                    \caption{\texttt{createPlayer()} method }
                \end{table*}

                \begin{table*}[!hp]
                    \centering
                    \begin{tabulary}{0.9\textwidth}{|p{8em}|p{8em}|p{28em}|}
                        \hline
                        \multicolumn{3}{|l|}{ \texttt{createTournament ( name, max\_rounds )}}\\
                        \hline
                        \textbf{Parameter} & \textbf{Data Type} & \textbf{Descriptor}\\
                        \hline
                        \texttt{name} & \texttt{String} & name of the tournament.\\
                        \hline
                        \texttt{max\_rounds} & \texttt{Integer} & the maximum number of rounds in the tournament.\\
                        \hline
                        \multicolumn{3}{|l|}{creates a tournament with the given \texttt{name} and \texttt{max\_rounds}}\\
                        \hline
                        \multicolumn{3}{|l|}{\textbf{Returns:}}\\
                        \hline
                        \multicolumn{3}{|l|}{\texttt{\{"outcome" : True\}} on success.}\\
                        \hline
                        \multicolumn{3}{|l|}{\texttt{\{"outcome" : False, "reason" : \dots\}} on failure.}\\
                        \hline
                    \end{tabulary}
                    \caption{\texttt{createTournament()} method }
                \end{table*}

                \begin{table*}[!hp]
                    \centering
                    \begin{tabulary}{0.9\textwidth}{|p{8em}|p{8em}|p{28em}|}
                        \hline
                        \multicolumn{3}{|l|}{ \texttt{finishRound ( r\_id )}}\\
                        \hline
                        \textbf{Parameter} & \textbf{Data Type} & \textbf{Descriptor}\\
                        \hline
                        \texttt{r\_id} & \texttt{Integer} & the round ID.\\
                        \hline
                        \multicolumn{3}{|l|}{sets the round as completed.}\\
                        \hline
                        \multicolumn{3}{|l|}{\textbf{Returns:}}\\
                        \hline
                        \multicolumn{3}{|l|}{\texttt{\{"outcome" : True\}} on success.}\\
                        \hline
                        \multicolumn{3}{|l|}{\texttt{\{"outcome" : False, "reason" : \dots\}} on failure.}\\
                        \hline
                    \end{tabulary}
                    \caption{\texttt{finishRound()} method }
                \end{table*}

                \begin{table*}[!hp]
                    \centering
                    \begin{tabulary}{0.9\textwidth}{|p{8em}|p{8em}|p{28em}|}
                        \hline
                        \multicolumn{3}{|l|}{ \texttt{generatePairing ( t\_id )}}\\
                        \hline
                        \textbf{Parameter} & \textbf{Data Type} & \textbf{Descriptor}\\
                        \hline
                        \texttt{t\_id} & \texttt{Integer} & the tournament ID.\\
                        \hline
                        \multicolumn{3}{|l|}{generates a round and the matches of the round.}\\
                        \hline
                        \multicolumn{3}{|l|}{\textbf{Returns:}}\\
                        \hline
                        \multicolumn{3}{|l|}{\texttt{\{"outcome" : True\}} on success.}\\
                        \hline
                        \multicolumn{3}{|l|}{\texttt{\{"outcome" : False, "reason" : \dots\}} on failure.}\\
                        \hline
                    \end{tabulary}
                    \caption{\texttt{generatePairing()} method }
                \end{table*}

                \begin{table*}[!hp]
                    \centering
                    \begin{tabulary}{0.9\textwidth}{|p{8em}|p{8em}|p{28em}|}
                        \hline
                        \multicolumn{3}{|l|}{ \texttt{getPlayers ( p\_id )}}\\
                        \hline
                        \textbf{Parameter} & \textbf{Data Type} & \textbf{Descriptor}\\
                        \hline
                        \texttt{p\_id} & \texttt{Integer} & the id of the player.\\
                        \hline
                        \multicolumn{3}{|l|}{returns the player object of the given \texttt{p\_id}}\\
                        \hline
                        \multicolumn{3}{|l|}{\textbf{Returns:}}\\
                        \hline
                        \multicolumn{3}{|l|}{\texttt{\{"outcome" : True, "rows": \{\dots\}\}} on success. \texttt{"rows"} contains player attributes and the paired}\\
                        \multicolumn{3}{|l|}{values for the specified player.}\\
                        \hline
                        \multicolumn{3}{|l|}{\texttt{\{"outcome" : False, "reason" : \dots\}} on failure.}\\
                        \hline
                    \end{tabulary}
                    \caption{\texttt{getPlayer()} method }
                \end{table*}

                \begin{table*}[!hp]
                    \centering
                    \begin{tabulary}{0.9\textwidth}{|p{8em}|p{8em}|p{28em}|}
                        \hline
                        \multicolumn{3}{|l|}{ \texttt{listActiveTournamentPlayers ( t\_id )}}\\
                        \hline
                        \textbf{Parameter} & \textbf{Data Type} & \textbf{Descriptor}\\
                        \hline
                        \texttt{t\_id} & \texttt{Integer} & the tournament ID.\\
                        \hline
                        \multicolumn{3}{|l|}{returns a list of player objects that are currently participating in the tournament.}\\
                        \hline
                        \multicolumn{3}{|l|}{\textbf{Returns:}}\\
                        \hline
                        \multicolumn{3}{|l|}{\texttt{\{"outcome" : True, "rows": [ \{\dots\}, \{\dots\}, \dots, \{\dots\} ]\}} on success. \texttt{"rows"} contains }\\
                        \multicolumn{3}{|l|}{player attributes and the paired values. Returns a list of dictionaries.}\\
                        \hline
                        \multicolumn{3}{|l|}{\texttt{\{"outcome" : False, "reason" : \dots\}} on failure.}\\
                        \hline
                    \end{tabulary}
                    \caption{\texttt{listActiveTournamentPlayers()} method }
                \end{table*}

                \begin{table*}[!hp]
                    \centering
                    \begin{tabulary}{0.9\textwidth}{|p{8em}|p{8em}|p{28em}|}
                        \hline
                        \multicolumn{3}{|l|}{ \texttt{listPlayers ( )}}\\
                        \hline
                        \textbf{Parameter} & \textbf{Data Type} & \textbf{Descriptor}\\
                        \hline
                        \texttt{----} & \texttt{----} & \texttt{----}\\
                        \hline
                        \multicolumn{3}{|l|}{returns a list of player objects that exist in the system.}\\
                        \hline
                        \multicolumn{3}{|l|}{\textbf{Returns:}}\\
                        \hline
                        \multicolumn{3}{|l|}{\texttt{\{"outcome" : True, "rows": [ \{\dots\}, \{\dots\}, \dots, \{\dots\} ]\}} on success. \texttt{"rows"} contains }\\
                        \multicolumn{3}{|l|}{player attributes and the paired values. Returns a list of dictionaries.}\\
                        \hline
                        \multicolumn{3}{|l|}{\texttt{\{"outcome" : False, "reason" : \dots\}} on failure.}\\
                        \hline
                    \end{tabulary}
                    \caption{\texttt{listPlayers()} method }
                \end{table*}

                \begin{table*}[!hp]
                    \centering
                    \begin{tabulary}{0.9\textwidth}{|p{8em}|p{8em}|p{28em}|}
                        \hline
                        \multicolumn{3}{|l|}{ \texttt{listTournament ( sort\_on, filter\_types )}}\\
                        \hline
                        \textbf{Parameter} & \textbf{Data Type} & \textbf{Descriptor}\\
                        \hline
                        \texttt{sort\_on} & \texttt{List} & a list of attributes to sort on.\\
                        \hline
                        \texttt{filter\_types} & \texttt{Character} & flags to filter the returning list by. Flags consist of waiting, started and finished.\\
                        \hline
                        \multicolumn{3}{|l|}{returns a list of tournaments objects with defined attributes}\\
                        \hline
                        \multicolumn{3}{|l|}{\textbf{Returns:}}\\
                        \hline
                        \multicolumn{3}{|l|}{\texttt{\{"outcome" : True, "rows": [ \{\dots\}, \{\dots\}, \dots, \{\dots\} ]\}} on success. \texttt{"rows"} contains }\\
                        \multicolumn{3}{|l|}{tournament attributes and the paired values. Returns a list of dictionaries.}\\
                        \hline
                        \multicolumn{3}{|l|}{\texttt{\{"outcome" : False, "reason" : \dots\}} on failure.}\\
                        \hline
                    \end{tabulary}
                    \caption{\texttt{listTournament()} method }
                \end{table*}

                \begin{table*}[!hp]
                    \centering
                    \begin{tabulary}{0.9\textwidth}{|p{8em}|p{8em}|p{28em}|}
                        \hline
                        \multicolumn{3}{|l|}{ \texttt{listTournamentPlayers ( t\_id )}}\\
                        \hline
                        \textbf{Parameter} & \textbf{Data Type} & \textbf{Descriptor}\\
                        \hline
                        \texttt{t\_id} & \texttt{Integer} & the tournament id.\\
                        \hline
                        \multicolumn{3}{|l|}{returns a list of tournaments players}\\
                        \hline
                        \multicolumn{3}{|l|}{\textbf{Returns:}}\\
                        \hline
                        \multicolumn{3}{|l|}{\texttt{\{"outcome" : True, "rows": [ \{\dots\}, \{\dots\}, \dots, \{\dots\} ]\}} on success. \texttt{"rows"} contains all }\\
                        \multicolumn{3}{|l|}{ players that are currently or in the past existed for that tournament. Returns a list of dictionaries.}\\
                        \hline
                        \multicolumn{3}{|l|}{\texttt{\{"outcome" : False, "reason" : \dots\}} on failure.}\\
                        \hline
                    \end{tabulary}
                    \caption{\texttt{listTournamentPlayers()} method }
                \end{table*}

                \begin{table*}[!hp]
                    \centering
                    \begin{tabulary}{0.9\textwidth}{|p{8em}|p{8em}|p{28em}|}
                        \hline
                        \multicolumn{3}{|l|}{ \texttt{matchList ( r\_id )}}\\
                        \hline
                        \textbf{Parameter} & \textbf{Data Type} & \textbf{Descriptor}\\
                        \hline
                        \texttt{r\_id} & \texttt{Integer} & the round ID.\\
                        \hline
                        \multicolumn{3}{|l|}{returns a list of matches for the current round..}\\
                        \hline
                        \multicolumn{3}{|l|}{\textbf{Returns:}}\\
                        \hline
                        \multicolumn{3}{|l|}{\texttt{\{"outcome" : True, "rows": [ \{\dots\}, \{\dots\}, \dots, \{\dots\} ]\}} on success. \texttt{"rows"} contains }\\
                        \multicolumn{3}{|l|}{match attributes and the paired values. Returns a list of dictionaries.}\\
                        \hline
                        \multicolumn{3}{|l|}{\texttt{\{"outcome" : False, "reason" : \dots\}} on failure.}\\
                        \hline
                    \end{tabulary}
                    \caption{\texttt{matchList()} method }
                \end{table*}

                \begin{table*}[!hp]
                    \centering
                    \begin{tabulary}{0.9\textwidth}{|p{8em}|p{8em}|p{28em}|}
                        \hline
                        \multicolumn{3}{|l|}{ \texttt{removePlayer ( p\_id, t\_id )}}\\
                        \hline
                        \textbf{Parameter} & \textbf{Data Type} & \textbf{Descriptor}\\
                        \hline
                        \texttt{p\_id} & \texttt{Integer} & the player id.\\
                        \hline
                        \texttt{t\_id} & \texttt{Integer} & the tournament ID.\\
                        \hline
                        \multicolumn{3}{|l|}{adds the player of \texttt{p\_id} to the tournament \texttt{t\_id}}\\
                        \hline
                        \multicolumn{3}{|l|}{\textbf{Returns:}}\\
                        \hline
                        \multicolumn{3}{|l|}{\texttt{\{"outcome" : True\}} on success.}\\
                        \hline
                        \multicolumn{3}{|l|}{\texttt{\{"outcome" : False, "reason" : \dots\}} on failure.}\\
                        \hline
                    \end{tabulary}
                    \caption{\texttt{removePlayer()} method }
                \end{table*}

                \begin{table*}[!hp]
                    \centering
                    \begin{tabulary}{0.9\textwidth}{|p{8em}|p{8em}|p{28em}|}
                        \hline
                        \multicolumn{3}{|l|}{ \texttt{roundList ( t\_id )}}\\
                        \hline
                        \textbf{Parameter} & \textbf{Data Type} & \textbf{Descriptor}\\
                        \hline
                        \texttt{t\_id} & \texttt{Integer} & the tournament ID.\\
                        \hline
                        \multicolumn{3}{|l|}{gets the list of rounds part of the tournament.}\\
                        \hline
                        \multicolumn{3}{|l|}{\textbf{Returns:}}\\
                        \hline
                        \multicolumn{3}{|l|}{\texttt{\{"outcome" : True\}} on success.}\\
                        \hline
                        \multicolumn{3}{|l|}{\texttt{\{"outcome" : False, "reason" : \dots\}} on failure.}\\
                        \hline
                    \end{tabulary}
                    \caption{\texttt{roundList()} method }
                \end{table*}

                \begin{table*}[!hp]
                    \centering
                    \begin{tabulary}{0.9\textwidth}{|p{8em}|p{8em}|p{28em}|}
                        \hline
                        \multicolumn{3}{|l|}{ \texttt{searchPlayers ( partial\_name )}}\\
                        \hline
                        \textbf{Parameter} & \textbf{Data Type} & \textbf{Descriptor}\\
                        \hline
                        \texttt{partial\_name} & \texttt{String} & partial string name.\\
                        \hline
                        \multicolumn{3}{|l|}{searches for a player with the given \texttt{partial\_name} and returns players name}\\
                        \hline
                        \multicolumn{3}{|l|}{\textbf{Returns:}}\\
                        \hline
                        \multicolumn{3}{|l|}{\texttt{\{"outcome" : True, "rows" : [ \{\dots\}, \{\dots\}, \dots, \{\dots\} ]\}} on success. \texttt{"rows"} contains a list}\\
                        \multicolumn{3}{|l|}{of dictionaries, containing player with names that match the given string.}\\
                        \hline
                        \multicolumn{3}{|l|}{\texttt{\{"outcome" : False, "reason" : \dots\}} on failure.}\\
                        \hline
                    \end{tabulary}
                    \caption{\texttt{searchPlayers()} method }
                \end{table*}

                \begin{table*}[!hp]
                    \centering
                    \begin{tabulary}{0.9\textwidth}{|p{8em}|p{8em}|p{28em}|}
                        \hline
                        \multicolumn{3}{|l|}{ \texttt{setMatchTournament ( m\_id, p1\_wins, p2\_wins, draws )}}\\
                        \hline
                        \textbf{Parameter} & \textbf{Data Type} & \textbf{Descriptor}\\
                        \hline
                        \texttt{m\_id} & \texttt{Integer} & the match id.\\
                        \hline
                        \texttt{p1\_wins} & \texttt{Integer} & player 1's wins.\\
                        \hline
                        \texttt{p2\_wins} & \texttt{Integer} & player 2's wins.\\
                        \hline
                        \texttt{draws} & \texttt{Integer} & number of draws.\\
                        \hline
                        \multicolumn{3}{|l|}{sets the scoring for the match between player 1 and player 2.}\\
                        \hline
                        \multicolumn{3}{|l|}{\textbf{Returns:}}\\
                        \hline
                        \multicolumn{3}{|l|}{\texttt{\{"outcome" : True\}} on success.}\\
                        \hline
                        \multicolumn{3}{|l|}{\texttt{\{"outcome" : False, "reason" : \dots\}} on failure.}\\
                        \hline
                    \end{tabulary}
                    \caption{\texttt{setMatchTournament()} method }
                \end{table*}
                \begin{table*}[!hp]
                    \centering
                    \begin{tabulary}{0.9\textwidth}{|p{8em}|p{8em}|p{28em}|}
                        \hline
                        \multicolumn{3}{|l|}{ \texttt{startTournament ( t\_id )}}\\
                        \hline
                        \textbf{Parameter} & \textbf{Data Type} & \textbf{Descriptor}\\
                        \hline
                        \texttt{t\_id} & \texttt{Integer} & the tournament ID.\\
                        \hline
                        \multicolumn{3}{|l|}{starts the specified tournament. This enables processing of the tournament.}\\
                        \hline
                        \multicolumn{3}{|l|}{\textbf{Returns:}}\\
                        \hline
                        \multicolumn{3}{|l|}{\texttt{\{"outcome" : True\}} on success.}\\
                        \hline
                        \multicolumn{3}{|l|}{\texttt{\{"outcome" : False, "reason" : \dots\}} on failure.}\\
                        \hline
                    \end{tabulary}
                    \caption{\texttt{startTournament()} method }
                \end{table*}

        \newpage

        \subsection{GUI Implementation}
            \subsubsection{GUI Modules}
                \begin{figure}[H]
                    \fbox{\includegraphics[width=\textwidth]{HLR-GUI_State_Trans}}
                    \caption{GUI Interaction Diagram}
                \end{figure}
                \newpage
            \subsubsection{GUI Module}
                The GUI has been logically divided into modules. Communication between each module is through a interface that stores the current state, allowing other states and modules to interact with the stored state.
                \begin{table*}[h!]
                    \centering
                    \begin{tabulary}{\textwidth}{|p{12em}|p{32em}|}
                        \hline
                        \textbf{State}                      & \textbf{Description} \\
                        \hline
                        \texttt{Start}                      & Initial entry. This is the initial entry window that is displayed on execution.\\
                        \hline
                        \texttt{End}                        & Terminal Exit. This results the application terminating.\\
                        \hline
                        \texttt{Match Module}               & Handles match updates and interactions.\\
                        \hline
                        \texttt{Player Module}              & Handles player creations, removals, and interactions.\\
                        \hline
                        \texttt{Round Module}               & Handles round interactions.\\
                        \hline
                        \texttt{Tournament Module}          & Handles tournament creations, selection, and interactions.\\
                        \hline
                        \end{tabulary}
                    \caption{GUI Interaction Description}
                \end{table*}
    \newpage

    \section{User Guide}
        The following section introduces the necessary steps to install and initialize the environment and application. This has also been dictated in the \texttt{readme}.
        \subsection{Development Environment Configuration and Usage}
            \subsubsection{Database Server Configuration}
                A public server has been set up for easy access and configuration.
                \newline
                \begin{table*}[!hp]
                    \centering
                    \begin{tabulary}{\textwidth}{|c|}
                        \hline
                        \texttt{ttdbserver.student.rit.edu:5000}\\
                        \hline
                    \end{tabulary}
                    \caption{\texttt{Public Server Connection}}
                \end{table*}
            \subsubsection{Environment Installation}
                The following section defines all necessary steps and requirements to executing the application.
                \newline
                The following list defines required software that must be installed by the user before continuing.
                \begin{table*}[!hp]
                    \centering
                    \begin{tabulary}{\textwidth}{|c|}
                        \hline
                        \texttt{Software}\\
                        \hline
                        \texttt{\href{https://virtualenv.pypa.io/en/stable/installation/}{Virtual Environment}}\\
                        \hline
                        \texttt{\href{https://docs.python.org/3/library/tkinter.html}{TKinter 8.6}}\\
                        \hline
                        \texttt{\href{https://www.python.org/downloads/release/python-350/}{Python 3.5}}\\
                        \hline
                    \end{tabulary}
                    \caption{\texttt{Required Environment Software}}
                \end{table*}
                \newline
                Once the required software has been installed, one may refer to the \texttt{readme} for the specific installation process of the environment. Windows and Linux machines will have slightly different instructions.
            \subsubsection{Environment Configuration}
                Database connection can be configured within \texttt{config.json}. The configuration must be configured prior to usage.

        \subsection{Application Usage}
            The following section introduces the necessary steps to run the application. This has also been dictated in the \texttt{readme}.
            \newline
            Two separate applications are provided, a console-based application (for development and testing) and a graphic user interface. Both must be run with \texttt{virtual environment}.
            \subsubsection{Console Application}
            The console application can be executed through the terminal or terminal-variant with the following command.
                \begin{lstlisting}[language=sh, caption=client console application usage]
                python ./src/client.py
                \end{lstlisting}
            \subsubsection{GUI Application}
            The GUI application can be executed through the terminal or terminal-variant with the following command.
                \begin{lstlisting}[language=sh,caption=client GUI application usage ]
                python ./src/clientGUI.py
                \end{lstlisting}
            \subsubsection{Application States}

        \subsection{Console Application Manual}
            The following section introduces operating and handling the application.
            \subsubsection{Commands}
                All operations can be viewed by using the \texttt{help} command.
        \newpage

        \subsection{GUI Interaction Overview}
            The following section describes application interaction.
            \subsubsection{GUI Interaction Overview}
                \begin{table*}[h!]
                    \centering
                    \begin{tabulary}{\textwidth}{|p{12em}|p{32em}|}
                        \hline
                        \textbf{State}                      & \textbf{Description} \\
                        \hline
                        \texttt{Start}                      & Initial entry. This is the initial entry window that is displayed on execution.\\
                        \hline
                        \texttt{End}                        & Terminal Exit. This results the application terminating.\\
                        \hline
                        \texttt{Player Creation}            & Player creation.\\
                        \hline
                        \texttt{Tournament Selection}       & Select an active tournament.\\
                        \texttt{Tournament Creation}        & Creates a new tournament.\\
                        \hline
                        \texttt{Add Player}                 & Adds the player(s) to the tournament.\\
                        \hline
                        \texttt{Remove Player}              & Remove the player(s) to the tournament.\\
                        \hline
                        \texttt{Round Selection}            & Select a round to view.\\
                        \hline
                        \texttt{Finish Round}               & Finish the round.\\
                        \hline
                        \texttt{Match Selection}            & Select a match for update.\\
                        \hline
                        \texttt{Update Match}               & Update a match.\\
                        \hline
                        \end{tabulary}
                    \caption{GUI Interaction Description}
                \end{table*}
                \newpage
            \subsubsection{GUI Usage}
                \begin{figure}[H]
                    \fbox{\includegraphics[height=21cm]{HLR-GUI_Interactions}}
                    \centering
                    \caption{GUI Interaction Diagram}
                \end{figure}
                \newpage

    \section{Appendices}
        \subsection{Figures}
            \listoffigures

        \newpage

        \subsection{Tables}
            \listoftables

        \newpage

        \subsection{Listings}
            \lstlistoflistings
        % \subsection{Index}
            % \printindex
\end{document}
