\documentclass[11pt]{article}

\usepackage[margin=1in]{geometry}
\usepackage{verbatim}
\usepackage{array}
\usepackage{parskip}
\usepackage{graphicx}
\graphicspath{ {images/} }

\setlength\parskip{\baselineskip}
\setlength\parindent{0pt}
\newcommand{\mtg}{\textit{Magic$\colon$ the Gathering\textsuperscript{TM}} }

\begin{document}
    \title{\mtg Tournament Recorder}
    \author{Team Tangent Tally \\ Benjamin Maitland \\ Jonathan Lo \\ Richard Ditullio \\ William Frey}
    \date{\today}
    
    \begin{titlepage}
        \maketitle
    \end{titlepage}
    
    \section{Overview}
    This application allows Tournament coordinators to record and track tournament data for a typical \mtg tournament.
    
    \subsection{Team Tangent Tally Members}
    \begin{center}
        \begin{tabular}{|l|l|}
            \hline
            \textbf{Roles} & \textbf{Name}\\
            \hline
            Team Leader & Benjamin Maitland\\
            \hline
            Design Engineer & Jonathan Lo\\
            \hline
            Test Engineer & Richard Ditullio\\
            \hline
            UI Engineer & William Frey\\
            \hline
        \end{tabular}
    \end{center}
    
    \section{Introduction}
    This section will discuss generic and detailed information about \mtg tournaments. The following definitions will be utilized later in the document.
    \subsection{Tournament Overview}
    \mtg tournaments are run as a multiple round Swiss tournament with a cut to a single elimination top 8. Pairing for each round is determined by player's current standings, where players with similar standings are paired to play against each other. At the end of a certain number of rounds, usually determined by the tournament's size, the top eight players by standing take part in a single-elimination finals bracket. 
    
    
    \subsection{Definitions}
    
    \begin{center}
        \begin{tabular}{|m{4cm}|m{11cm}|}
            \hline
            \textbf{Terms} & \textbf{Definitions} \\
            \hline
            Game & A single play of the game \mtg.\\
            \hline
            Match & A series of games between two players. A winner and loser are defined at the end of the match, unless in cases of a draw.\\
            \hline
            Round & A collection of matches.\\
            \hline
            Tournament & A collection of rounds, declaring a single winner.\\
            \hline
            Match Winner & The player that is first to win two games in a match.\\
            \hline
            Match Pairing & The pairing between two explicit players\\
            \hline
            Match Score & The points given to the winner and loser, or in the case of a draw, both players.\\
            \hline
            Bye & A match with one player. The player is automatically declared the winner. A player can only play in one bye per tournament.\\
            \hline
            Player Standing & A measurement of a player's comparative score amongst other players in the current tournament.\\
            \hline
        \end{tabular}
        \label{table:1}
    \end{center}
    
    \section{Requirements}
    \subsection{Software Requirements}
    \begin{itemize}
        \item The application shall be executable from a Windows or Linux operating system.
        \item The application shall be implemented using Python 3.5.
        \item The application shall utilize a MySQL-compatible Database to store, update, and access tournament records.
        \item The application shall implement the standard \mtg tournament structure and procedures.
    \end{itemize}
    
    \subsection{Software Design Requirements}
    \begin{itemize}
        \item The application shall implement an appropriate and usable interface for users.
        \item The application shall interface with the database, accessing and retrieving live tournament data and records.
        \begin{itemize}
            \item The application shall store player information, such as: name, unique identifier, matches, and tournament standings.
            \item The application shall store tournament information, such as: name, and number of initial players.
            \item The application shall store tournament round information, such as: matches in the round, and round results.
            \item The application shall store tournament match information, such as: players involved in each match and match results.
        \end{itemize}
        \item The application shall compute the necessary pairings for each round.
        \item The application shall require users to update necessary data, upon completion of a major state, such as match scores.
        \item The application shall compute necessary logic utilizing updated data within the database before an explicit state, such as match pairing for the next round.
        \item The application shall be testable with an automated and/or manual test(s).
        \item The application shall utilize appropriate design patterns, such as the Model-View-Controller design pattern.
    \end{itemize}
    
    \subsection{Software Architecture Design}
    \begin{figure}[h]
        \fbox{\includegraphics[width=\textwidth]{HLR-State_Trans}}
        \centering
        \caption{State Transition Diagram}
    \end{figure}
    
\end{document}
