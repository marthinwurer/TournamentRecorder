\documentclass[11pt]{article}

\usepackage[margin=1in]{geometry}
\usepackage{verbatim}
\usepackage{array}
\usepackage{parskip}
\setlength\parskip{\baselineskip}
\setlength\parindent{0pt}

\begin{document}
\title{Tournament Game Recorder}
\author{Team Tangent Tally & Richard Ditullio & William Frey & Jonathan Lo & Benjamin Maitland }
\date{\today}

\maketitle
\newpage

\section{Overview}
This application allows Tournament coordinators to record and track tournament data for a typical \textit{Magic: the Gathering} tournament match pairing and player game recording.

\section{Introduction}
This section will discuss generic and detailed information about \textit{Magic the Gathering} tournaments. The following definitions and acronyms will be utilized later in the document.
\subsection{Tournament Overview}
Explain how a tournament works
\subsection{Acronyms}
Define any acronyms
\subsection{Definitions}

\begin{center}
\begin{tabular}{|m{4cm}|m{11cm}|}
    \hline
    \textbf{Terms} & \textbf{Definitions} \\
    \hline
    Match Pairing & The pairing between two explicit players\\
    \hline
    Match Score & The points given to the winner and loser, or in cases of a draw, the two inclusive players.\\
    \hline
    Tournament & A collection of rounds, declaring a single winner.\\
    \hline
    Rounds & A collection of matches, declaring a number of winners calculated from the number of matches.\\
    \hline
    Matches & A series of games between two players. A winner and loser is defined at the end of the match, unless of cases of a draw. Matches contain three games.\\
    \hline
    Match Winner & The first player that is first to win two out of three games in a match.\\
    \hline
    Game & A single play of a \textit{Magic the Gathering} game.\\
    \hline
    Bye & A match with one player. The player is automatically declared the winner.\\
    \hline
    Player Standing & A measurement of a player's comparative score amongst other players.\\
    \hline
\end{tabular}
\label{table:1}
\end{center}

\section{Software Requirements}
\begin{itemize}
\item The application shall be executable from a Windows or Linux operating system.
\item The application shall be implemented using Python 3.5.
\item The application shall be utilizing MySQL Database.
\item The application shall implement an interface for users to interact.
\item The application shall be testable, either automated or manual.
\item The application shall maintain and record relevant tournament data.
\item The application shall implement algorithms for required logic.
\end{itemize}

\section{Software Design}

\section{Software Implementation}

\section{Appendix}
\subsection{Reference Documentation}
\subsection{Tables}
\listoftables

\end{document}
